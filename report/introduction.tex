\chapter{Introduction}

In the context of Centrale Lille's last year project, Alyssia Colas and I tackled 
the design of a soft wearable robotic orthosis for the elbow. This topic was proposed 
by Prof. Kruszewski, who is affiliated with the DEFROST team at Inria Lille. The team 
has secured funding for the development of this system. The goal is to 
provide an efficient low-cost and low-maintenance solution for elbow rehabilitation.  

\bigskip

The loss of motor function is commonly caused by neurological disorders such as 
stroke, cerebral palsy or Parkinson's disease. Rehabilitation therapy is essential 
to regain functional capacity in the impacted areas through neuropasticity. 
Traditionally, physical therapists would conduct one-on-one manual treatment 
to recovering patients. In recent years, researchers have proposed numerous 
robot-aided rehabilitation devices as a prospective solution. The most recent research 
explores soft wearable devices because they present a number of advantages 
compared to the currently more prevalent rigid exoskeletons. They present a more 
compact structure, lighter weight, lower inertia, safer operation, more comfortable 
interaction, and they are easier to adapt to anatomical variations.  

\bigskip

To create a novel solution to the problem, the project was split into two separate 
parts: the design of the mechanical system responsible for the kinematic properties 
of the solution, and the design of the control system responsible for the transformation 
of user intention into movement. The latter is the topic of this report paper and the 
subject of my work.  
