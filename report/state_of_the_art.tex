\chapter{State of the art}
Multiple research papers have covered the topic of soft wearable orthoses for 
upper as well as lower limbs. The work described in this report found most of its 
inspiration in the following research papers. 

First, Lu et al. \cite{lu_development_2019} developed 
a joint torque estimation control strategy for a soft elbow exoskeleton to 
provide effective power assistance. Later, Wu at al. 
have expanded on this work by optimizing anchor points in the mechanical design and 
modifying the control strategy \cite{wu_neural-network-enhanced_2019}, then by updating 
the estimation of joint torque to determine motion intention \cite{wu_adaptive_2023}.
Lastly, the recent works of Toro-Ossaba et al. \cite{toro-ossaba_myoelectric_2024} 
on the control of a soft elbow exoskeleton influenced numerous decisions made 
throughout this work.  


\section{Sensors}
When it comes to the control system itself, multiple sensors can be used to acquire 
relevant data. This data is required for the generation of input as well as the 
control feedback. In the case of an upper limb orthotic device, the system input 
is defined by the user's motion intention and feedback can include position, velocity, 
acceleration, torque and force.

According to a systematic review of the signals, sensors and methods for controlling 
active upper limb orthotic devices \cite{dos_santos_signals_2023}:
\begin{quotation}
The results indicate that electromyography (EMG) was the most used signal for 
controlling these devices, described in 36 articles, followed by 
electroencephalography (EEG), used by 15 authors. Among the other signals used 
were force myography (FMG), force-sensing resistors (FSR), inertial measurement 
unit (IMU) sensors and external torques.
\end{quotation}

\subsection{Motion intention}
Multiple signals can indicate motion intention in a patient. As previously stated, 
EMG and EEG are most often used for this application. Alternatives such as FMG may 
also be used.  
\begin{IEEEitemize}
\item EMG measures muscle response or electrical activity in response to a nerve's 
stimulation of the muscle
\item EEG measures electrical activity in the brain
\item FMG measures the volumetric changes of the underlying musculotendinous complex 
during muscle activities \cite{jiang_exploration_2017}
\end{IEEEitemize}
One advantage of EMG and EEG signals over FMG is that they manifest before the 
corresponding muscle response. Begovic et al. show that on average, the electromechanical 
delay between EMG and force is 50ms \cite{begovic_detection_2014}. Furthermore, 
EEG precede their EMG response by around 23ms \cite{xu_delay_2016}. Therefore, 
EMG and EEG signals can allow for a quicker reaction to the user's motion intention 
compared to FMG. 
\subsection{Kinematics}
In the case of rigid exoskeleton devices, incremental encoders can suffice to 
determine system kinematics. For soft devices, various sensors can serve the 
purpose of gathering this information. The most used sensor for the 
acquisition of joint angles is the inertial 
measurement unit (IMU). These sensors are used in multiple works \cite{lu_development_2019, 
wu_adaptive_2023}. Two IMUs, one on each side of the target joint, 
are needed to measure the joint angle and the measurement is made by performing quaternion 
calculations. 

Gibbs et al. \cite{gibbs_wearable_2005} present an alternative 
method of measuring joint angle consisting in wearable conductive fiber sensors. 
These sensors present an average root mean square (RMS) error of 2.5° when 
compared with a standard goniometer.  

A systematic review on the validity and reliability of wearable sensors for joint 
angle estimation \cite{poitras_validity_2019} encourages the use of IMUs as low-cost 
alternatives to state-of-the art motion capture systems.  

\subsection{Kinetics}
Multiple sensors may be used to acquire feedback of the system kinetics for the 
control of active orthotic devices: 

\begin{IEEEitemize}
  \item Sangha et al. \cite{sangha_compact_2016} used FSR to measure user applied 
    force for the control of a wrist orthosis  
  \item Lu et al. \cite{lu_development_2019} used a tension sensor 
\end{IEEEitemize}

Further research was not conducted on this type of feedback.  

\section{Actuators}
% hydraulics, motors, types of motors
In the conducted research, various actuators were found in the implementation 
of orthotic devices: 

\begin{IEEEitemize}
  \item Baysal et al. \cite{baysal_implementation_2023} used a pneumatic system
  \item Copaci et al. \cite{copaci_sma_2019} used a shape memory alloy (SMA) based 
    approach, by heating and cooling SMA cables to make them expand and retract
  \item Multiple papers \cite{lu_development_2019, wu_neural-network-enhanced_2019, 
    wu_adaptive_2023} use a DC motor and cable system
\end{IEEEitemize}

After discussing with Antoine Dequidt, expert in the design and control of 
robotic systems applied to robotic exoskeletons, rehabilitation and assistive 
robots, a DC motor based approach, almost identical to the ones cited above, 
was suggested. Furthermore, Dequidt provided recommendations for the motor 
specifications, namely: the motor should be brushless and it should have high 
nominal torque to prevent the need for a large reduction gear, thus ensuring 
that the system remains reversible.  

\FloatBarrier
\section{Algorithms}
The algorithms put in place for the control of active orthotic devices mainly 
serve the purpose of quantifying user motion intention. According to the 
systematic review mentioned above \cite{dos_santos_signals_2023}: 

\begin{quote}
  It was possible to observe a heterogeneity on the algorithms used for identifying 
  motion intention, such as neural networks, linear discriminant analysis 
  classifiers, support vector machines, proportional control and thresholding. 
  It was also identified the need for further studies comparing these techniques.
\end{quote}

However, in the conducted research, neural networks (NN) were the most prevalent 
approach, see e.g. \cite{lu_development_2019, wu_neural-network-enhanced_2019, 
wu_adaptive_2023, sangha_compact_2016}. 
