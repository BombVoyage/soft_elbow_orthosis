\chapter{State of the art}

\section{Introduction}
% All similar research
Multiple research papers have covered the topic of soft wearable orthoses for 
upper as well as lower limbs. Lu et al. \cite{lu_development_2019} have developed 
a joint torque estimation control strategy for a soft elbow exoskeleton to 
provide effective power assistance. Later, Wu at al. 
have expanded on this work by optimizing anchor points in the mechanical design and 
modifying the control strategy \cite{wu_neural-network-enhanced_2019}, then by updating 
the estimation of joint torque to determine motion intention \cite{wu_adaptive_2023}.


\section{Sensors}
% muscle activity, angle, force
When it comes to the control system itself, multiple sensors can be used to acquire 
relevant data. This data is required for the generation of input as well as the 
control feedback. In the case of an upper limb orthotic device, the system input 
is defined by the user's motion intention and feedback can include position, velocity, 
acceleration, torque and force.

According to a systematic review of the signals, sensors and methods for controlling 
active upper limb orthotic devices \cite{dos_santos_signals_2023}:
\begin{quotation}
The results indicate that electromyography (EMG) was the most used signal for 
controlling these devices, described in 36 articles, followed by 
electroencephalography (EEG), used by 15 authors. Among the other signals used 
were force myography (FMG), force-sensing resistors (FSR), inertial measurement 
unit (IMU) sensors and external torques.
\end{quotation}

\subsection{Motion intention}
Multiple signals can indicate motion intention in a patient. As previously stated, 
EMG and EEG are most often used for this application. Alternatives such as FMG may 
also be used.  
\begin{IEEEitemize}
\item EMG measures muscle response or electrical activity in response to a nerve's 
stimulation of the muscle
\item EEG measures electrical activity in the brain
\item FMG measures the volumetric changes of the underlying musculotendinous complex 
during muscle activities \cite{jiang_exploration_2017}
\end{IEEEitemize}
One advantage of EMG and EEG signals over FMG is that they manifest before the 
corresponding muscle response. Begovic et al. show that on average, the electromechanical 
delay between EMG and force is 50ms \cite{begovic_detection_2014}. Furthermore, 
EEG precede their EMG response by around 23ms \cite{xu_delay_2016}. Therefore, 
EMG and EEG signals can allow for a quicker reaction to the user's motion intention 
compared to FMG. 
\subsection{Kinematics}
Various sensors can serve the purpose of gathering information about the system 
kinematics. The most 
% IMUs: accel, gyro, magn

\subsection{Dynamics}

\section{Actuators}
% hydraulics, motors, types of motors

\section{Processing}
% How to get motor input from muscle activity
